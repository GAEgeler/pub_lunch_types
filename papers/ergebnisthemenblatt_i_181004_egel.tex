\documentclass[12pt,ngerman,]{article}
\usepackage{lmodern}
\usepackage{amssymb,amsmath}
\usepackage{ifxetex,ifluatex}
\usepackage{fixltx2e} % provides \textsubscript
\ifnum 0\ifxetex 1\fi\ifluatex 1\fi=0 % if pdftex
  \usepackage[T1]{fontenc}
  \usepackage[utf8]{inputenc}
\else % if luatex or xelatex
  \ifxetex
    \usepackage{mathspec}
  \else
    \usepackage{fontspec}
  \fi
  \defaultfontfeatures{Ligatures=TeX,Scale=MatchLowercase}
\fi
% use upquote if available, for straight quotes in verbatim environments
\IfFileExists{upquote.sty}{\usepackage{upquote}}{}
% use microtype if available
\IfFileExists{microtype.sty}{%
\usepackage{microtype}
\UseMicrotypeSet[protrusion]{basicmath} % disable protrusion for tt fonts
}{}
\usepackage[margin = 1.2in]{geometry}
\usepackage{hyperref}
\hypersetup{unicode=true,
            pdfauthor={Gian-Andrea Egeler, Priska Baur},
            pdfborder={0 0 0},
            breaklinks=true}
\urlstyle{same}  % don't use monospace font for urls
\ifnum 0\ifxetex 1\fi\ifluatex 1\fi=0 % if pdftex
  \usepackage[shorthands=off,main=ngerman]{babel}
\else
  \usepackage{polyglossia}
  \setmainlanguage[]{german}
\fi
\usepackage{graphicx,grffile}
\makeatletter
\def\maxwidth{\ifdim\Gin@nat@width>\linewidth\linewidth\else\Gin@nat@width\fi}
\def\maxheight{\ifdim\Gin@nat@height>\textheight\textheight\else\Gin@nat@height\fi}
\makeatother
% Scale images if necessary, so that they will not overflow the page
% margins by default, and it is still possible to overwrite the defaults
% using explicit options in \includegraphics[width, height, ...]{}
\setkeys{Gin}{width=\maxwidth,height=\maxheight,keepaspectratio}
\IfFileExists{parskip.sty}{%
\usepackage{parskip}
}{% else
\setlength{\parindent}{0pt}
\setlength{\parskip}{6pt plus 2pt minus 1pt}
}
\setlength{\emergencystretch}{3em}  % prevent overfull lines
\providecommand{\tightlist}{%
  \setlength{\itemsep}{0pt}\setlength{\parskip}{0pt}}
\setcounter{secnumdepth}{0}
% Redefines (sub)paragraphs to behave more like sections
\ifx\paragraph\undefined\else
\let\oldparagraph\paragraph
\renewcommand{\paragraph}[1]{\oldparagraph{#1}\mbox{}}
\fi
\ifx\subparagraph\undefined\else
\let\oldsubparagraph\subparagraph
\renewcommand{\subparagraph}[1]{\oldsubparagraph{#1}\mbox{}}
\fi

%%% Use protect on footnotes to avoid problems with footnotes in titles
\let\rmarkdownfootnote\footnote%
\def\footnote{\protect\rmarkdownfootnote}

%%% Change title format to be more compact
\usepackage{titling}

% Create subtitle command for use in maketitle
\newcommand{\subtitle}[1]{
  \posttitle{
    \begin{center}\large#1\end{center}
    }
}

\setlength{\droptitle}{-2em}

  \title{Ergebnisthemenblätter NOVANIMAL\\
Menü-Linien, ein veraltetes Konzept? Wie Gäste dazu bewegt werden können
vermehrt ressourcenleichte Gerichte zu wählen}
    \pretitle{\vspace{\droptitle}\centering\huge}
  \posttitle{\par}
    \author{Gian-Andrea Egeler, Priska Baur}
    \preauthor{\centering\large\emph}
  \postauthor{\par}
      \predate{\centering\large\emph}
  \postdate{\par}
    \date{Oktober 2018}

\usepackage{fancyhdr}
\usepackage{setspace}
\onehalfspacing
\usepackage{float}
\usepackage{graphicx}
\usepackage[font=small,labelfont=bf]{caption}

\begin{document}
\maketitle

\hypertarget{lead-teaser}{%
\subsubsection{1. Lead / Teaser}\label{lead-teaser}}

Die zentrale Bedeutung tierischer Produkte spiegelt sich im Angebot der
Gastronomie. Im Jahr 2015 sind Fleischgerichte (20.3\%) mit den
entsprechenden Beilagen (30.8\%) im Zusammenhang der
Ausser-Haus-Verpflegung die meistkonsumierten Speisen in der Schweiz
(GastroSuisse
\protect\hyperlink{ref-gastrosuisse_branchenspiegel_2016}{2016}). Gemäss
der ersten in den Jahren 2014/2015 durchgeführten nationalen
Ernährungserhebung menuCH essen 70 \% der Bevölkerung zwischen 18 und 75
Jahren am Mittag auswärts (Bochud u.~a.
\protect\hyperlink{ref-bochud_anthropometric_2017}{2017}). Daher rückt
die Gastronomie als zentraler Akteur einer innovativen und nachhaltigen
Ernährungswirtschaft ins Blickfeld. Welche Innovationen in der
Gastronomie könnten dazu beitragen, den Pro-Kopf-Verbrauch an tierischen
Nahrungsmitteln zu senken?

\hypertarget{forschungsfragen-und-ziele}{%
\subsubsection{2. Forschungsfragen und
Ziele}\label{forschungsfragen-und-ziele}}

Die übergeordnete Forschungsfrage dieses Arbeitspackets WPIII.1 war es
die Gäste dazu zu bewegen, häufiger ressourcenschonende
vegetarische\footnote{vegetarisch = ovo-lakto-vegetarisch (inkl. Eier
  und Milch)} oder vegane\footnote{vegan = ausschliesslich pflanzliche
  Zustaten} Gerichte zu wählen. Die abgeleitete Forschungsfragen
interessiert es wie das Menü-Angebot oder die Menü-Beschriftung die
Menü-Wahl aller Mensabesucher beeinflusst?

\newpage

\hypertarget{theoretischer-hintergrund-und-methoden}{%
\subsubsection{3. Theoretischer Hintergrund und
Methoden}\label{theoretischer-hintergrund-und-methoden}}

Zusammen mit den Praxispartnern, einem Catering Unternehmen und dem
Facility Management einer Hochschule, wurde ein Quasi-Experiment
vorbereitet, das als Pilotversuch in zwei Hochschulmensen durchgeführt
wird. Konkret wird untersucht, wie die Gäste auf ein verändertes
Menü-Angebot mit einem höheren Anteil an ressourcenschonenden
vegetarischen oder veganen Gerichten reagieren.

\par

Das Quasi-Experiment findet im Herbstsemester (HS) 2017 während 12
Wochen (60 Tage) statt und bestand aus zwei Mensazyklen à 6 Wochen
(siehe Abbildung \ref{fig:fig1}). Über den gesamten
Untersuchungszeitraum werden insgesamt 93 verschiedene Gerichten
angeboten. In den 6 Referenz- bzw. Basiswochen wurden zwei fleisch- oder
fischhaltige Gerichte und ein vegetarisches Gericht angeboten. In den 6
Interventionswochen wurde das Verhältnis umgekehrt und es wurden ein
veganes, ein vegetarisches und ein fleisch- oder fischhaltiges Gericht
angeboten. Beim veganen Angebot wurde zwischen eigenständige authenische
Gerichte (z. B. Linsen-Dal) und vegane Gerichte mit Fleischsubstituten
(z. B. Quornragout) unterschieden. Basis- und Interventionsangebote
wechselten wöchentlich ab. Während der gesamten 12 Wochen konnten die
Gäste jeweils auf ein Buffet ausweichen und ihre Mahlzeit aus warmen und
kalten Komponenten selber zusammenstellen. Die nachfolgenden Analysen
und Ergebnisse basieren auf Daten aus den Kassensystemen des Catering
Unternehmens. Für die Analysen sind insgesamt mehr als 26'000
Transaktionen resp. Kauf eines Gerichts während der Mittagszeit
berücksichtigt worden.

\begin{figure}[H]

{\centering \includegraphics[width=0.8\linewidth]{design_eng_experiment_ohne datum_180426_03egel} 

}

\caption{\label{fig:fig1} Die Abbildung 1 zeigt das Versuchsdesign der ersten 6 Experimentwochen (Kalenderwochen 40 bis 45).}\label{fig:unnamed-chunk-1}
\end{figure}

\newpage

\hypertarget{ergebnisse}{%
\subsubsection{4. Ergebnisse}\label{ergebnisse}}

\hypertarget{was-wurde-angeboten}{%
\paragraph{4.1 Was wurde angeboten?}\label{was-wurde-angeboten}}

Von insgesamt 93 verschiedenen Gerichten, welche über zwei
Mensazyklen\footnote{ein Zyklus besteht aus 6 Wochen, danach wiederholen
  sich die Gerichten} verteilt angeboten wurden, enthielten 48 Prozent
(n = 45) der Gerichte Fleisch. Ein Drittel (n = 31) der Gerichte wurde
vegetarisch und die restlichen 17 Prozent (n = 16) ausschliesslich
pflanzlich angeboten.

\par

90 verschiedenen Mahlzeiten wurden pro Mensa und Mensazyklus auf drei
Menü-Linien (Favorite, Kitchen und World) angeboten. Wird das gesamte
Angebot (zwei Mensazyklen) auf beide Mensen und über alle Tage addiert,
ergibt das ein geplantes Angebot von 360 Gerichten\footnote{90 Gerichte
  x 2 Mensazyklen x 2 Mensen}. Aus betrieblichen Gründen kam es beim
Catering Unternehmen des Öfteren zu Zusatzangeboten, welche das geplante
Angebot zusätzlich vergrösserte. In den 12 Wochen wurden zu den total
geplanten 360 Gerichten noch 120 Gerichten zusätzlich angeboten (siehe
Abbildung \ref{fig:fig2}).

\begin{figure}[H]

{\centering \includegraphics[width=1\linewidth]{ergebnisthemenblatt_i_181004_egel_files/figure-latex/unnamed-chunk-3-1} 

}

\caption{\label{fig:fig2} Die Abbildung zeigt das geplante (480) und angebotene Angebot (600) für ein ganzes Herbstsemester (60 Tage) und zwei Hochschulmensen nach den fünf Menü-Inhalten}\label{fig:unnamed-chunk-3}
\end{figure}

Die Abbildung \ref{fig:fig2} zeigt für die Herbstemester die kalkulierte
Planung von 2015 und 2016 als auch das geplante Angebot des
Quasi-Experiments im HS 2017. Die dritte Säule zeigt die tatsächlich
angebotene Auswahl für das HS 2017. Das geplante Angebot und das
tatsächliche Angebot im HS 2017 unterscheiden sich in den
Menü-Inhalten\footnote{Fleisch oder Fisch, Vegetarisch, Vegan
  (authentisch) und Vegan (Fleischsubstitut)} nicht statisch
(\(\chi^2\)(3) = 1.589, \emph{p} = 0.662). Auch beim näheren Betrachten
unterschieden sich die geplanten und tatsächlichen Angeboten für das HS
2017 in den Menü-Inhalten zwischen den Basis- (\(\chi^2\)(3) = 7.143,
\emph{p} = 0.067) und Interventionswochen (\(\chi^2\)(3) = 5.984,
\emph{p} = 0.112) nicht.

\par

Wird das tatsächliche Angebot näher beleuchtet, wurden in über alle
Interventionenwochen (\emph{n} = 238) 40 Prozent fleischhaltige, knapp
einem Drittel vegetarische und 27 Prozent vegane Gerichte angeboten. In
den 12 Basiswochen (\emph{n} = 242) entiehlten \(\frac{2}{3}\) Prozent
des Angebots Fleisch oder Fisch. Vegetarische Gerichte machten rund 29
Prozent und vegane Gerichte knapp 1 Prozent aus.

\par

Im Vergleich zu den vergangenen Herbstsemestern 2015 \& 2016 wuchs das
vegetarische und vegane Angebot im Herbstsemester 2017 von 40 Prozent
auf knapp 47 Prozent, wobei das Hot and Cold-Buffet nicht als
vegetarisches Angebot mitgezählt wurde. \newpage

\hypertarget{was-wurde-verkauft}{%
\paragraph{4.2 Was wurde verkauft?}\label{was-wurde-verkauft}}

Die totalen Verkaufszahlen des Herbstsemesters 2017 haben sich im
Vergleich zu den zwei HS 2015 \& 2016 nur marginal verändert. Die
Abbildung \ref{fig:fig3} zeigt alle verkauften Gerichten nach
Menü-Inhalt (vegan, vegetarisch, Fleisch und Buffet), welche während des
Quasi-Experiments verkauft wurden.

Der Abbildung \ref{fig:fig3} ist zu entnehmen, dass die wöchentlichen
Verkaufszahlen sich statistisch nicht unterscheiden (\emph{F}(1,10) =
284.209, \emph{p} \textless{} .001). Des Weiteren gab es Unterschiede in
den Verkaufszahlen zwischen den Basis- und Interventionswochen
(\emph{H}(7) = 42.56, \emph{p} \textless{} .001). Neben den schwankenden
verkaufszahlen der Fleischgerichte, fällt es auf, dass vegane
authentische Gerichte deutlich besser verkauft wurden als vegan
fleischsubstitituierte Gerichte.

\newpage

\hypertarget{wurden-in-den-interventionswochen-auch-mehr-vegetarischen-gerichten-verkauft}{%
\paragraph{4.3 Wurden in den Interventionswochen auch mehr vegetarischen
Gerichten
verkauft?}\label{wurden-in-den-interventionswochen-auch-mehr-vegetarischen-gerichten-verkauft}}

Wird ein grösseres und vielfältigeres Angebot an attraktiven
vegetarischen und veganen Gerichten von den Mensabesuchern häufiger
gewählt? Die Abbildung \ref{fig:fig4} zeigt, dass es keine statistischen
Unterschiede in den Verkaufszahlen beim Hot and Cold und im
vegetarischen Angebot zwischen den beiden Bedingungen Basis \&
Intervention gab. Auffallend sind die Unterschiede im Verkauf von
Fleischgerichten, welche deutlich seltener in den Interventionswochen
als in den Basiswochen verkauft wurden. Dass sich mehr vegane Gerichte
in den Interventionswochen gekauft wurden lag daran, dass es in den
Basiswochen lediglich ein kleines veganes Angebot gab.

\begin{figure}[H]

{\centering \includegraphics{ergebnisthemenblatt_i_181004_egel_files/figure-latex/unnamed-chunk-6-1} 

}

\caption{\label{fig:fig4} Die Abbildung 4 zeigt die verkauften Gerichten nach Menü-Inahlt und Bedingung (Intervention- und Basiswochen)}\label{fig:unnamed-chunk-6}
\end{figure}

In der Abbildung \ref{fig:fig4} kann zudem entnommen werden, dass in den
Interventionswochen im Verkauf von veganen (authentisch) und veganen
(Fleischsubstitut) Gerichten Unterschiede gibt. Auf aggregierter Ebene
scheinen die authentischen veganen Gerichten sich deutlich besser zu
verkaufen als die veganen Gerichten mit Fleischsubstitute.

\hypertarget{lassen-sich-vegetarische-gerichte-auch-teuerer-verkaufen}{%
\paragraph{4.4 Lassen sich vegetarische Gerichte auch teuerer
verkaufen?}\label{lassen-sich-vegetarische-gerichte-auch-teuerer-verkaufen}}

Sind die aktuellen Menü-Linien tatsächlich ein veraltetes Konzept? Nebst
der Änderung im Angebot, war eine weitere Intervention die Gerichte über
neutrale Menü-Gefässe sog. Menü-Linien zu vertreiben. Für das
Quasi-Experiment wurde die Menü-Linie Green zu World unbenannt. Die
Menü-Inhalte wurden randomisiert über diese drei Menü-Linien angeboten.
Die Abbildung \ref{fig:fig5} vergleicht die Verkaufszahlen gemessen an
den Menü-Linien der Jahren 2015, 2016 und 2017.

\begin{figure}[H]

{\centering \includegraphics{ergebnisthemenblatt_i_181004_egel_files/figure-latex/unnamed-chunk-7-1} 

}

\caption{\label{fig:fig5} Die Abbildung 5 zeigt die verkauften Gerichten nach Menü-Linien im Vergleich (2015, 2016 und 2017)}\label{fig:unnamed-chunk-7}
\end{figure}

Die Abbildung \ref {fig:fig5} zeigt, dass es zwischen den drei
Herbstsemestern klare Unterschiede in den Verkauszahlen zwischen den
Menü-Linien gab. Die Grafik zwar lässt keine Schlüsse über deren
Menü-Inhalt zu, aber die randomisierte Ausgabe der Gerichte konnte
aufzeigen, dass es einen Einfluss auf die Menü-Wahl genommen hat. So
wurden auch vegetarischen Gerichte auf der teureren Menü-Linie «Kitchen»
problemlos von den Gästen gekauft.

\newpage

\hypertarget{diskussion}{%
\subsubsection{5. Diskussion}\label{diskussion}}

Nach unserem Wissen wurde noch kein Real-Labor-Experiment durchgeführt,
welches die Reaktion der Gäste auf ein ressourcenschonenderes Angebot
aufzeichnete. Es gibt einige Untersuchungen wobei mittels
Quasi-Experimenten getestet wurde wie die Mensagäste auf bestimmte
Labels (z. B. zu gesunder Ernährung oder umweltfreundlichen Gerichten)
(Hoefkens u.~a. \protect\hyperlink{ref-hoefkens_what_2012}{2012};
Spaargaren u.~a. \protect\hyperlink{ref-spaargaren_consumer_2013}{2013};
Thorndike u.~a. \protect\hyperlink{ref-thorndike_2-phase_2012}{2012}),
auf Präsentationen oder Positionierung von Nahrungsmitteln (Wansink und
Hanks \protect\hyperlink{ref-wansink_slim_2013}{2013}; Kleef, Otten, und
Trijp \protect\hyperlink{ref-van_kleef_healthy_2012}{2012}) oder auf
soziale oder injuktive Normen (Collins u.~a.
\protect\hyperlink{ref-collins_two_2019}{2019}) reagierten.

\par

Zusammengefasst zeigen die Ergebnisse dieser Studie, dass eine Erhöhung
des vegetarischen und veganen Angebots in Kombination mit neutralen
Menü-Lienien und einer randomisierten Ausgabe des Menü-Inahlts zu einer
Senkung des Fleischkonsums führen könnten. Ob nun die randomisierte
Ausgabe, die neutralen Menü-Linien oder die quantitative Erhöhung von
vegetarische und vegane Gerichten für diesen Effekt zuständig ist,
bedarf es an weiteren Untersuchungen. Ein weiterer Einflussfaktor könnte
das vergrösserte Angebot, welches durch das Caternig Unternehmen
generiert wurde, darstellen. Weiter konnte in diesem Quasi-Experiment
die qualitative Gleichwertigkeit der täglich angebotenen Gerichte nicht
überprüft werden, was gegebenenfalls bei dem Kauf eines Gerichts einen
Einfluss nehmen könnte. Zudem basieren die gezeigten Ergebnisse auf
aggregierte Verkaufszahlen.

\par

Gemäss früheren Studien (siehe Blanchette und Brug
\protect\hyperlink{ref-blanchette_determinants_2005}{2005};
Neumark-sztainer u.~a.
\protect\hyperlink{ref-neumark-sztainer_factors_1999}{1999}; Kleef,
Broek, und Trijp \protect\hyperlink{ref-van_kleef_exploiting_2015}{2015}
evtl noch buch aufführen von @rozin\_1\_2007) führt eine höhere
Verfügbarkeit, z. B. von gesundem Essen, auch zu einem Einfluss auf das
Ernährungsverhalten. Auch in neuren Studien (Ensaff u.~a.
\protect\hyperlink{ref-ensaff_food_2015}{2015}) scheint die
Verfügbarkeit als Nudging-Strategie eine Veränderung im Kaufverhalten
von pflanzenbasierten Nahrungsmitteln hervorzurufen. Auf unser
Quasi-Experiment angewendet, konnte eine Erhöhung von
ressourcenleichteren Gerichten resp. eine erhöhte Verfügbarkeit davon,
einen ähnlichen Effekt hervorrufen, dass mehr vegetarische und vegane
Gerichte resp. weniger fleischhaltige Gerichte verkauft wurden. Eine
weitere Erklärung unserer Erbegnissen könnte die Erhöhung der Auswahl
von Menü-Inhalten sein. Aschemann-Witzel und Kollegen
(\protect\hyperlink{ref-aschemann-witzel_effects_2013}{2013}) fanden
heraus, dass eine Erhöhung der Auswahlmenge einen Einfluss auf das
gesunde Wahlverhalten nimmt. Eine Kaufentscheidung zu verstehen und
demnach zu beeinflussen ist sehr komplex, denn wir fällen wir über 200
Essensentscheide pro Tag (Wansink und Sobal
\protect\hyperlink{ref-wansink_mindless_2007}{2007}). Es zeigt sich aber
immer mehr, dass die meisten Entscheide anhand einfachen Heuristiken
sog. Faustregeln gefällt werden (siehe Scheibehenne, Miesler, und Todd
\protect\hyperlink{ref-scheibehenne_fast_2007}{2007}). Inwiefern die
Intervention auch tatsächlich einen Einfluss auf das Wahl-Verhalten
nimmt, wird in einem separaten Ergebnisthemenblatt untersucht (siehe
Egeler \& Baur, 2018b).

\par

Ein weiteres spannendes Ergebniss dieses Quasi-Experiment war, dass in
den Interventionswochen die veganen authentische Gerichte sich besser
verkauften, als die veganen Gerichte mit Fleischsubstituten. Frühere
Studien zeigten, dass der Gerichte-Kontext (z. B. Ingredienzen) einen
starken Einfluss auf den Konsum von Fleischsubstituten nimmt (Elzerman
u.~a. \protect\hyperlink{ref-elzerman_consumer_2011}{2011}). Zudem ist
die Akzeptanz von Fleischsubstituten immernoch relativ klein (siehe Hoek
\protect\hyperlink{ref-hoek_will_2010}{2010}; Boer und Aiking
\protect\hyperlink{ref-de_boer_merits_2011}{2011}).

\par

Obwohl die Menü-Linien resp. Menü-Labels oftmals für die Gästeführung
eingeführt werden, konnte im Quasi-Experiment aufgezeigt werden, dass
durch die Wahl von neutralen Menü-Linien und eine randomisierte Ausgabe
der Gerichte auch teurere vegetarische und vegane Gerichte von den
Mensabesucher gekauft werden. Auch in der Literatur wird kontrovers
disskutiert, ob Labels überhaupt einen Einfluss auf die Menü-Wahl nehmen
(siehe Spaargaren u.~a.
\protect\hyperlink{ref-spaargaren_consumer_2013}{2013}; Kiszko u.~a.
\protect\hyperlink{ref-kiszko_influence_2014}{2014}; Aschemann-Witzel
u.~a. \protect\hyperlink{ref-aschemann-witzel_effects_2013}{2013}).

\hypertarget{schlussfolgerungen}{%
\subsubsection{6. Schlussfolgerungen}\label{schlussfolgerungen}}

Eine Erhöhung von Anzahl und Anteil an vegetarischen und veganen
Gerichten in Kombination mit neutralen Menü-Linien und randomisierter
Ausgabe der drei Menü-Inhalten (Fleisch oder Fisch, vegetarisch und
vegan), scheint den Konsum von Fleischgerichten zu reduzieren. An diesem
Punkt ist es wichtig zu erwähnen, dass die vegetarischen und veganen
Gerichte diskret neben anderen Inhaltsstoffen deklarieren und nicht
angepriesen werden sollten. Um den Gast vermehrt dazu führen, sich mit
dem Menü-Inahlt auseinanderzusetzten, alte Gewohnheiten fallen zu lassen
und Neophobie zu verkleienrn, sollten die Gerichte unabhängig von
Menü-Inahlt randomisierung über alle Menü-Linien anbegoten werden.

\par

Wenn es um den einzelnen Verkauf von Gerichten geht, scheinen
authentische Gerichte besser als Gerichte mit Fleischsubstitute
abzuschneiden. Daher nicht nur Fleischsubstitute anbieten und wenn
Fleischsubstitute angeboten werden, sollte auf den Menü-Kontext geachten
werden. Zudem zeigte sich, dass auch vegetarische und vegane Gerichte
auf der teureren Menü-Linie ``Kitchen'' ebenfalls von den Gästen
akzeptiert und gekauft wurden.

\par

In diesem Quasi-Experiment konnte nicht die qualitative Gleichwertigkeit
der angebotenen Gerichte berücksichtigt werden. Wird zum Beispiel ein
Fleischburger mit Pommes und ein vegetarisches Linsen-Dahl von den
Mensabesuchern resp. von Koch-Experten als gleichwertig wahrgenommen?
Bei einem weiteren Mensa-Experiment sollte die Gleichwertigkeit der
Gerichte bei der Angebotsentwicklung mittels Fachleuten oder
Koch-Experten berücksichtigt werden.

\par

Mit diesem Quasi-Experiment konnte gezeigt werden, dass sich die
Verkaufszahlen über die Zeit durch eine Erhöhung des vegetarischen und
veganen Angebots nicht zurückgegangen sind. Das speziellen Setting und
die spezielle Stichprobe haben bestimmt dafür beigetragen,
nichtsdestotrotz wären weitere Pilotstudien an anderen Kantinen mit
anderen Stichproben resp. Mensabesuchern wünschenswert.

Weitere Informationen zum Projekt können auf unserer
\href{novanimal.ch}{Webpage} gefunden werden.

\hypertarget{danksagung}{%
\subsubsection{7. Danksagung}\label{danksagung}}

Wir bedanken uns bei der SV Schweiz und insbesondere bei den konkreten
Verantwortlichen (Area- und Restaurantmanager und Küchenteam) für die
hilfsbereite und unkomplizierte Unterstützung (Lieferung von Daten,
Rezepturen, Beantwortung von Fragen).

\hypertarget{hinweise-des-autors}{%
\subsubsection{8. Hinweise des Autors}\label{hinweise-des-autors}}

Alle Skripte für diesen Bericht sind auf
\href{https://github.com/GAEgeler/tilldata_2017}{github} verfügbar.

\hypertarget{quellen}{%
\subsubsection*{9. Quellen}\label{quellen}}
\addcontentsline{toc}{subsubsection}{9. Quellen}

\hypertarget{refs}{}
\leavevmode\hypertarget{ref-aschemann-witzel_effects_2013}{}%
Aschemann-Witzel, Jessica, Klaus G. Grunert, Hans C. M. van Trijp,
Svetlana Bialkova, Monique M. Raats, Charo Hodgkins, Grazyna
Wasowicz-Kirylo, und Joerg Koenigstorfer. 2013. „Effects of nutrition
label format and product assortment on the healthfulness of food
choice``. \emph{Appetite} 71 (Dezember):63--74.
\url{https://doi.org/10.1016/j.appet.2013.07.004}.

\leavevmode\hypertarget{ref-blanchette_determinants_2005}{}%
Blanchette, L., und J. Brug. 2005. „Determinants of fruit and vegetable
consumption among 6--12-year-old children and effective interventions to
increase consumption``. \emph{Journal of Human Nutrition and Dietetics}
18 (6):431--43. \url{https://doi.org/10.1111/j.1365-277X.2005.00648.x}.

\leavevmode\hypertarget{ref-bochud_anthropometric_2017}{}%
Bochud, Murielle, Angéline Chatelan, Juan-Manuel Blanco, und Sigrid
Beer-Borst. 2017. „Anthropometric characteristics and indicators of
eating and physical activity behaviors in the Swiss adult population``.
\url{https://www.blv.admin.ch/dam/blv/de/dokumente/lebensmittel-und-ernaehrung/ernaehrung/menuch-bericht.pdf.download.pdf/menuch-bericht.pdf}.

\leavevmode\hypertarget{ref-de_boer_merits_2011}{}%
Boer, Joop de, und Harry Aiking. 2011. „On the merits of plant-based
proteins for global food security: Marrying macro and micro
perspectives``. \emph{Ecological Economics}, Special Section: Ecological
Economics and Environmental History, 70 (7):1259--65.
\url{https://doi.org/10.1016/j.ecolecon.2011.03.001}.

\leavevmode\hypertarget{ref-collins_two_2019}{}%
Collins, Emily I. M., Jason M. Thomas, Eric Robinson, Paul Aveyard,
Susan A. Jebb, C. Peter Herman, und Suzanne Higgs. 2019. „Two
observational studies examining the effect of a social norm and a health
message on the purchase of vegetables in student canteen settings``.
\emph{Appetite} 132 (Januar):122--30.
\url{https://doi.org/10.1016/j.appet.2018.09.024}.

\leavevmode\hypertarget{ref-elzerman_consumer_2011}{}%
Elzerman, Johanna E., Annet C. Hoek, Martinus A. J. S. van Boekel, und
Pieternel A. Luning. 2011. „Consumer acceptance and appropriateness of
meat substitutes in a meal context``. \emph{Food Quality and Preference}
22 (3):233--40. \url{https://doi.org/10.1016/j.foodqual.2010.10.006}.

\leavevmode\hypertarget{ref-ensaff_food_2015}{}%
Ensaff, Hannah, Matt Homer, Pinki Sahota, Debbie Braybrook, Susan Coan,
Helen McLeod, Hannah Ensaff, u.~a. 2015. „Food Choice Architecture: An
Intervention in a Secondary School and its Impact on Students'
Plant-based Food Choices``. \emph{Nutrients} 7 (6):4426--37.
\url{https://doi.org/10.3390/nu7064426}.

\leavevmode\hypertarget{ref-gastrosuisse_branchenspiegel_2016}{}%
GastroSuisse. 2016. „Branchenspiegel 2016: Entwicklung von Angebot und
Nachfrage``. Powerpoint.

\leavevmode\hypertarget{ref-hoefkens_what_2012}{}%
Hoefkens, Christine, Prakashan Chellattan Veettil, Guido Van
Huylenbroeck, John Van Camp, und Wim Verbeke. 2012. „What nutrition
label to use in a catering environment? A discrete choice experiment``.
\emph{Food Policy} 37 (6):741--50.
\url{https://doi.org/10.1016/j.foodpol.2012.08.004}.

\leavevmode\hypertarget{ref-hoek_will_2010}{}%
Hoek, Annet C. 2010. \emph{Will novel protein foods beat meat? Consumer
acceptance of meat substitutes ; a multidisciplinary research approach}.

\leavevmode\hypertarget{ref-kiszko_influence_2014}{}%
Kiszko, Kamila M., Olivia D. Martinez, Courtney Abrams, und Brian Elbel.
2014. „The Influence of Calorie Labeling on Food Orders and Consumption:
A Review of the Literature``. \emph{Journal of Community Health} 39
(6):1248--69. \url{https://doi.org/10.1007/s10900-014-9876-0}.

\leavevmode\hypertarget{ref-van_kleef_exploiting_2015}{}%
Kleef, Ellen van, Oriana van den Broek, und Hans C.M. van Trijp. 2015.
„Exploiting the Spur of the Moment to Enhance Healthy Consumption:
Verbal Prompting to Increase Fruit Choices in a Self-Service
Restaurant``. \emph{Applied Psychology: Health and Well-Being} 7
(2):149--66. \url{https://doi.org/10.1111/aphw.12042}.

\leavevmode\hypertarget{ref-van_kleef_healthy_2012}{}%
Kleef, Ellen van, Kai Otten, und Hans CM van Trijp. 2012. „Healthy
snacks at the checkout counter: A lab and field study on the impact of
shelf arrangement and assortment structure on consumer choices``.
\emph{BMC Public Health} 12 (1):1072.
\url{https://doi.org/10.1186/1471-2458-12-1072}.

\leavevmode\hypertarget{ref-neumark-sztainer_factors_1999}{}%
Neumark-sztainer, DIANNE, MARY Story, CHERYL Perry, und MARY ANNE Casey.
1999. „Factors Influencing Food Choices of Adolescents: Findings from
Focus-Group Discussions with Adolescents``. \emph{Journal of the
American Dietetic Association} 99 (8):929--37.
\url{https://doi.org/10.1016/S0002-8223(99)00222-9}.

\leavevmode\hypertarget{ref-rozin_1_2007}{}%
Rozin, P. 2007. „1 - Food choice: an introduction``. In
\emph{Understanding Consumers of Food Products}, herausgegeben von Lynn
Frewer und Hans van Trijp, 3--29. Woodhead Publishing Series in Food
Science, Technology and Nutrition. Woodhead Publishing.
\url{https://doi.org/10.1533/9781845692506.1.3}.

\leavevmode\hypertarget{ref-scheibehenne_fast_2007}{}%
Scheibehenne, Benjamin, Linda Miesler, und Peter M. Todd. 2007. „Fast
and frugal food choices: Uncovering individual decision heuristics``.
\emph{Appetite} 49 (3):578--89.
\url{https://doi.org/10.1016/j.appet.2007.03.224}.

\leavevmode\hypertarget{ref-spaargaren_consumer_2013}{}%
Spaargaren, Gert, C. S. A. (Kris) van Koppen, Anke M. Janssen, Astrid
Hendriksen, und Corine J. Kolfschoten. 2013. „Consumer Responses to the
Carbon Labelling of Food: A Real Life Experiment in a Canteen
Practice``. \emph{Sociologia Ruralis} 53 (4):432--53.
\url{https://doi.org/10.1111/soru.12009}.

\leavevmode\hypertarget{ref-thorndike_2-phase_2012}{}%
Thorndike, Anne N., Lillian Sonnenberg, Jason Riis, Susan Barraclough,
und Douglas E. Levy. 2012. „A 2-Phase Labeling and Choice Architecture
Intervention to Improve Healthy Food and Beverage Choices``.
\emph{American Journal of Public Health} 102 (3):527--33.
\url{https://doi.org/10.2105/AJPH.2011.300391}.

\leavevmode\hypertarget{ref-wansink_slim_2013}{}%
Wansink, Brian, und Andrew S. Hanks. 2013. „Slim by Design: Serving
Healthy Foods First in Buffet Lines Improves Overall Meal Selection``.
Herausgegeben von Xiaoxi Zhuang. \emph{PLoS ONE} 8 (10):e77055.
\url{https://doi.org/10.1371/journal.pone.0077055}.

\leavevmode\hypertarget{ref-wansink_mindless_2007}{}%
Wansink, Brian, und Jeffery Sobal. 2007. „Mindless Eating: The 200 Daily
Food Decisions We Overlook``. \emph{Environment and Behavior} 39
(1):106--23. \url{https://doi.org/10.1177/0013916506295573}.


\end{document}
