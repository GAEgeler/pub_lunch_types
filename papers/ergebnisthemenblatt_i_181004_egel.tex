\documentclass[12pt,ngerman,]{article}
\usepackage{lmodern}
\usepackage{amssymb,amsmath}
\usepackage{ifxetex,ifluatex}
\usepackage{fixltx2e} % provides \textsubscript
\ifnum 0\ifxetex 1\fi\ifluatex 1\fi=0 % if pdftex
  \usepackage[T1]{fontenc}
  \usepackage[utf8]{inputenc}
\else % if luatex or xelatex
  \ifxetex
    \usepackage{mathspec}
  \else
    \usepackage{fontspec}
  \fi
  \defaultfontfeatures{Ligatures=TeX,Scale=MatchLowercase}
\fi
% use upquote if available, for straight quotes in verbatim environments
\IfFileExists{upquote.sty}{\usepackage{upquote}}{}
% use microtype if available
\IfFileExists{microtype.sty}{%
\usepackage{microtype}
\UseMicrotypeSet[protrusion]{basicmath} % disable protrusion for tt fonts
}{}
\usepackage[margin = 1.2in]{geometry}
\usepackage{hyperref}
\hypersetup{unicode=true,
            pdfauthor={Gian-Andrea Egeler, Priska Baur},
            pdfborder={0 0 0},
            breaklinks=true}
\urlstyle{same}  % don't use monospace font for urls
\ifnum 0\ifxetex 1\fi\ifluatex 1\fi=0 % if pdftex
  \usepackage[shorthands=off,main=ngerman]{babel}
\else
  \usepackage{polyglossia}
  \setmainlanguage[]{german}
\fi
\usepackage{graphicx,grffile}
\makeatletter
\def\maxwidth{\ifdim\Gin@nat@width>\linewidth\linewidth\else\Gin@nat@width\fi}
\def\maxheight{\ifdim\Gin@nat@height>\textheight\textheight\else\Gin@nat@height\fi}
\makeatother
% Scale images if necessary, so that they will not overflow the page
% margins by default, and it is still possible to overwrite the defaults
% using explicit options in \includegraphics[width, height, ...]{}
\setkeys{Gin}{width=\maxwidth,height=\maxheight,keepaspectratio}
\IfFileExists{parskip.sty}{%
\usepackage{parskip}
}{% else
\setlength{\parindent}{0pt}
\setlength{\parskip}{6pt plus 2pt minus 1pt}
}
\setlength{\emergencystretch}{3em}  % prevent overfull lines
\providecommand{\tightlist}{%
  \setlength{\itemsep}{0pt}\setlength{\parskip}{0pt}}
\setcounter{secnumdepth}{0}
% Redefines (sub)paragraphs to behave more like sections
\ifx\paragraph\undefined\else
\let\oldparagraph\paragraph
\renewcommand{\paragraph}[1]{\oldparagraph{#1}\mbox{}}
\fi
\ifx\subparagraph\undefined\else
\let\oldsubparagraph\subparagraph
\renewcommand{\subparagraph}[1]{\oldsubparagraph{#1}\mbox{}}
\fi

%%% Use protect on footnotes to avoid problems with footnotes in titles
\let\rmarkdownfootnote\footnote%
\def\footnote{\protect\rmarkdownfootnote}

%%% Change title format to be more compact
\usepackage{titling}

% Create subtitle command for use in maketitle
\newcommand{\subtitle}[1]{
  \posttitle{
    \begin{center}\large#1\end{center}
    }
}

\setlength{\droptitle}{-2em}

  \title{Ergebnisthemenblätter NOVANIMAL\\
Menü-Linien, ein veraltetes Konzept? Wie Gäste dazu bewegt werden können
vermehrt ressourcenleichte Gerichte zu wählen}
    \pretitle{\vspace{\droptitle}\centering\huge}
  \posttitle{\par}
    \author{Gian-Andrea Egeler, Priska Baur}
    \preauthor{\centering\large\emph}
  \postauthor{\par}
      \predate{\centering\large\emph}
  \postdate{\par}
    \date{Oktober 2018}

\usepackage{fancyhdr}
\usepackage{placeins}
\usepackage{setspace}
\usepackage{chngcntr}
\onehalfspacing

\begin{document}
\maketitle

\hypertarget{lead-teaser}{%
\subsubsection{1. Lead / Teaser}\label{lead-teaser}}

Die zentrale Bedeutung tierischer Produkte spiegelt sich im Angebot der
Gastronomie. Im Jahr 2015 sind Fleischgerichte (20.3\%) mit den
entsprechenden Beilagen (30.8\%) im Zusammenhang der
Ausser-Haus-Verpflegung die meistkonsumierten Speisen in der Schweiz
(GastroSuisse 2016). Gemäss der ersten in den Jahren 2014/2015
durchgeführten nationalen Ernährungserhebung menuCH essen 70 \% der
Bevölkerung zwischen 18 und 75 Jahren am Mittag auswärts (Bochud u.~a.
2017). Daher rückt die Gastronomie als zentraler Akteur einer
innovativen und nachhaltigen Ernährungswirtschaft ins Blickfeld. Welche
Innovationen in der Gastronomie könnten dazu beitragen, den
Pro-Kopf-Verbrauch an tierischen Nahrungsmitteln zu senken?

\hypertarget{forschungsfragen-und-ziele}{%
\subsubsection{2. Forschungsfragen und
Ziele}\label{forschungsfragen-und-ziele}}

Die übergeordnete Forschungsfrage dieses Arbeitspackets WPIII.1 war es
die Gäste dazu zu bewegen, häufiger ressourcenschonende
vegetarische\footnote{vegetarisch = ovo-lakto-vegetarisch (inkl. Eier
  und Milch)} oder vegane\footnote{vegan = ausschliesslich pflanzliche
  Zustaten} Gerichte zu wählen. Die abgeleitete Forschungsfragen
interessiert es wie das Menü-Angebot oder die Menü-Beschriftung die
aggregierte Menü-Wahl beeinflusst?

\hypertarget{theoretischer-hintergrund-und-methoden}{%
\subsubsection{3. Theoretischer Hintergrund und
Methoden}\label{theoretischer-hintergrund-und-methoden}}

Zusammen mit den Praxispartnern, einem Catering Unternehmen und dem
Facility Management einer Hochschule, wurde ein Quasi-Experiment
vorbereitet, das als Pilotversuch in zwei Hochschulmensen durchgeführt
wird. Konkret wird untersucht, wie die Gäste auf ein verändertes
Menü-Angebot mit einem höheren Anteil an ressourcenschonenden
vegetarischen oder veganen Gerichten reagieren. Das Quasi-Experiment
findet im Herbstsemester 2017 während 12 Wochen (60 Tage) statt und
bestand aus zwei Mensazyklen à 6 Wochen (siehe Abbildung
\ref{fig:fig1}). Über den gesamten Untersuchungszeitraum werden
insgesamt 93 verschiedene Gerichten angeboten. In den 6 Referenz- bzw.
Basiswochen wurden zwei fleisch- oder fischhaltige Gerichte und ein
vegetarisches Gericht angeboten. In den 6 Interventionswochen wurde das
Verhältnis umgekehrt und es wurden ein veganes, ein vegetarisches und
ein fleisch- oder fischhaltiges Gericht angeboten. Beim veganen Angebot
wurde zwischen eigenständige authenische Gerichte (z. B. Linsen-Dal) und
vegane Gerichte mit Fleischsubstituten (z. B. Quornragout)
unterschieden. Basis- und Interventionsangebote wechselten wöchentlich
ab. Während der gesamten 12 Wochen konnten die Gäste jeweils auf ein
Buffet ausweichen und ihre Mahlzeit aus warmen und kalten Komponenten
selber zusammenstellen.

\begin{figure}
\includegraphics[width=1\linewidth]{design_eng_experiment_ohne datum_180426_03egel} \caption{\label{fig:fig1} Die Abbildung 1 zeigt das Versuchsdesign der ersten 6 Experimentwochen (Kalenderwochen 40 bis 45).}\label{fig:unnamed-chunk-1}
\end{figure}

\hypertarget{ergebnisse}{%
\subsubsection{4. Ergebnisse}\label{ergebnisse}}

\hypertarget{section}{%
\paragraph{4.1}\label{section}}

Wird ein grösseres und vielfältigeres Angebot an attraktiven
vegetarischen und veganen Gerichten von den Mensabesuchern häufiger
gewählt? Von insgesamt 93 verschiedenen Gerichten, welche über zwei
Mensazyklen verteilt angeboten wurden, enthielten 0 Prozent (n = 45) der
Gerichte Fleisch. Ein Drittel (n = 31) der Gerichte wurde vegetarisch
und die restlichen 0 Prozent (n = 16) ausschliesslich pflanzlich
angeboten. Das Angebot in den Interventionenwochen war zu einem Drittel
fleischhaltige, zu einem Drittel vegetarische und zu einem Drittel
vegane Gerichte. In den Basiswochen waren 50 Prozent des Angebots
fleischhaltige Gerichte. Vegetarische Gerichte machten rund 22 Prozent
und vegane Gerichte knapp 3 Prozent aus. Extrapoliert man das Angebot
auf beide Mensen und über alle Tage gibt das ein geplantes Angebot von
480 Gerichten\footnote{60 Tage x 2 Mensen x 4 Menü-Inhalten}. Aus
betrieblichen Gründen kam es beim des Catering Unternehmen zu
Zusatzangeboten, welche das Angebot grundsätzlich vergrösserten. Von den
geplanten 480 (pro Mensa) sind noch 120 Gerichten zusätzlich angeboten
worden. Das Verhätnis blieb jedoch1im Anhang leichten Abweichungen
hinsichtlich des ursprünglichen Versuchsdesigns. Die Abweichungen
beeinflussten aber nicht das gesamte Angebot.

Im Vergleich zu den vergangenen Jahren (2015 und 2016) wuchs das
vegetarische und vegane Angebot im 2017 von 30 Prozent auf knapp 40
Prozent, wobei das Hot and Cold-Buffet nicht als vegetarisches Angebot
mitgezählt wurde. In der Abbildung \ref{fig:fig2} zeigt alle verkauften
Gerichte nach Menü-Inhalt (vegan, vegetarisch, Fleisch und Buffet),
welche während des Quasi-Experiments angeboten wurden.

\begin{figure}
\centering
\includegraphics{ergebnisthemenblatt_i_181004_egel_files/figure-latex/unnamed-chunk-2-1.pdf}
\caption{\label{fig:fig2} Die Abbildung 2 zeigt alle verkauften
Gerichten (N = 26'234) aufgeteilt nach den fünf Menü-Inhalten über die
12 Wochen}
\end{figure}

\begin{verbatim}
FALSE Dunn (1964) Kruskal-Wallis multiple comparison
\end{verbatim}

\begin{verbatim}
FALSE   p-values adjusted with the Benjamini-Hochberg method.
\end{verbatim}

Der Abbildung \ref{fig:fig2} ist zu entnehmen, dass die wöchentlichen
Verkaufszahlen sich statistisch nicht unterscheiden (\emph{F}(1,10) =
284.209, \emph{p} = 0). Des Weiteren gab es Unterschiede in den
Verkaufszahlen zwischen den Interventions- und Basiswochen (\emph{H}(7)
= 42.56, \emph{p} \textless{} 0.001).

\begin{figure}
\centering
\includegraphics{ergebnisthemenblatt_i_181004_egel_files/figure-latex/unnamed-chunk-4-1.pdf}
\caption{\label{fig:fig3} Die Abbildung 3 zeigt die verkauften Gerichten
nach Menü-Inahlt und Bedingung (Intervention- und Basiswochen)}
\end{figure}

Die Abbildung \ref{fig:fig3} zeigt, dass es keine statistischen
Unterschiede in den Verkaufszahlen beim Hot and Cold und im
vegetarischen Angebot zwischen den beiden Bedingungen gab (Basis- und
Intervention). Auffallend sind die Unterschiede im Verkauf von
Fleischgerichten, welche deutlich seltener in den Interventionswochen
als in den Basiswochen verkauft wurden. Dass sich mehr vegane Gerichte
in den Interventionswochen gekauft wurden lag daran, dass es in den
Basiswochen ein kleines veganes Angebot gab.

Sind die aktuellen Menü-Linien tatsächlich ein veraltetes Konzept? Nebst
der Änderung im Angebot, war eine weitere Intervention die Gerichte über
neutrale Menü-Gefässe sog. Menü-Linien zu vertreiben. Für das
Quasi-Experiment wurde die Menü-Linie Green zu World unbenannt. Die
Menü-Inhalte wurden randomisiert über diese drei Menü-Linien angeboten.
Die Abbildung \ref{fig:fig4} vergleicht die Verkaufszahlen gemessen an
den Menü-Linien über die Jahren 2015, 2016 und 2017.

\begin{figure}
\centering
\includegraphics{ergebnisthemenblatt_i_181004_egel_files/figure-latex/unnamed-chunk-5-1.pdf}
\caption{\label{fig:fig4} Die Abbildung 4 zeigt die verkauften Gerichten
nach Menü-Linien im Vergleich (2015, 2016 und 2017)}
\end{figure}

Die Abbildung \ref {fig:fig4}

\hypertarget{diskussion-inkl.-einordnung-der-ergebnisse-in-die-internationale-forschung}{%
\subsubsection{5. Diskussion (inkl. Einordnung der Ergebnisse in die
internationale
Forschung)}\label{diskussion-inkl.-einordnung-der-ergebnisse-in-die-internationale-forschung}}

\hypertarget{schlussfolgerungen-mit-empfehlungenimplikationen-fur-praxis-und-politik}{%
\subsubsection{6. Schlussfolgerungen mit Empfehlungen/Implikationen für
Praxis und
Politik}\label{schlussfolgerungen-mit-empfehlungenimplikationen-fur-praxis-und-politik}}

\hypertarget{danksagung}{%
\subsubsection{7. Danksagung}\label{danksagung}}

Wir bedanken uns bei der SV Schweiz und insbesondere bei den konkreten
Verantwortlichen (Area- und Restaurantmanager und Küchenteam) für die
hilfsbereite und unkomplizierte Unterstützung (Lieferung von Daten,
Rezepturen, Beantwortung von Fragen).

\hypertarget{quellen}{%
\subsubsection*{8. Quellen}\label{quellen}}
\addcontentsline{toc}{subsubsection}{8. Quellen}

\hypertarget{refs}{}
\leavevmode\hypertarget{ref-bochud_anthropometric_2017}{}%
Bochud, Murielle, Angéline Chatelan, Juan-Manuel Blanco, und Sigrid
Beer-Borst. 2017. „Anthropometric characteristics and indicators of
eating and physical activity behaviors in the Swiss adult population``.
\url{https://www.blv.admin.ch/dam/blv/de/dokumente/lebensmittel-und-ernaehrung/ernaehrung/menuch-bericht.pdf.download.pdf/menuch-bericht.pdf}.

\leavevmode\hypertarget{ref-gastrosuisse_branchenspiegel_2016}{}%
GastroSuisse. 2016. „Branchenspiegel 2016: Entwicklung von Angebot und
Nachfrage``. Powerpoint.


\end{document}
