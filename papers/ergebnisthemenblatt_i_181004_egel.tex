\documentclass[12pt,ngerman,]{article}
\usepackage{lmodern}
\usepackage{amssymb,amsmath}
\usepackage{ifxetex,ifluatex}
\usepackage{fixltx2e} % provides \textsubscript
\ifnum 0\ifxetex 1\fi\ifluatex 1\fi=0 % if pdftex
  \usepackage[T1]{fontenc}
  \usepackage[utf8]{inputenc}
\else % if luatex or xelatex
  \ifxetex
    \usepackage{mathspec}
  \else
    \usepackage{fontspec}
  \fi
  \defaultfontfeatures{Ligatures=TeX,Scale=MatchLowercase}
\fi
% use upquote if available, for straight quotes in verbatim environments
\IfFileExists{upquote.sty}{\usepackage{upquote}}{}
% use microtype if available
\IfFileExists{microtype.sty}{%
\usepackage{microtype}
\UseMicrotypeSet[protrusion]{basicmath} % disable protrusion for tt fonts
}{}
\usepackage[margin=1in]{geometry}
\usepackage{hyperref}
\hypersetup{unicode=true,
            pdfauthor={Gian-Andrea Egeler, Priska Baur},
            pdfborder={0 0 0},
            breaklinks=true}
\urlstyle{same}  % don't use monospace font for urls
\ifnum 0\ifxetex 1\fi\ifluatex 1\fi=0 % if pdftex
  \usepackage[shorthands=off,main=ngerman]{babel}
\else
  \usepackage{polyglossia}
  \setmainlanguage[]{german}
\fi
\usepackage{graphicx,grffile}
\makeatletter
\def\maxwidth{\ifdim\Gin@nat@width>\linewidth\linewidth\else\Gin@nat@width\fi}
\def\maxheight{\ifdim\Gin@nat@height>\textheight\textheight\else\Gin@nat@height\fi}
\makeatother
% Scale images if necessary, so that they will not overflow the page
% margins by default, and it is still possible to overwrite the defaults
% using explicit options in \includegraphics[width, height, ...]{}
\setkeys{Gin}{width=\maxwidth,height=\maxheight,keepaspectratio}
\IfFileExists{parskip.sty}{%
\usepackage{parskip}
}{% else
\setlength{\parindent}{0pt}
\setlength{\parskip}{6pt plus 2pt minus 1pt}
}
\setlength{\emergencystretch}{3em}  % prevent overfull lines
\providecommand{\tightlist}{%
  \setlength{\itemsep}{0pt}\setlength{\parskip}{0pt}}
\setcounter{secnumdepth}{0}
% Redefines (sub)paragraphs to behave more like sections
\ifx\paragraph\undefined\else
\let\oldparagraph\paragraph
\renewcommand{\paragraph}[1]{\oldparagraph{#1}\mbox{}}
\fi
\ifx\subparagraph\undefined\else
\let\oldsubparagraph\subparagraph
\renewcommand{\subparagraph}[1]{\oldsubparagraph{#1}\mbox{}}
\fi

%%% Use protect on footnotes to avoid problems with footnotes in titles
\let\rmarkdownfootnote\footnote%
\def\footnote{\protect\rmarkdownfootnote}

%%% Change title format to be more compact
\usepackage{titling}

% Create subtitle command for use in maketitle
\newcommand{\subtitle}[1]{
  \posttitle{
    \begin{center}\large#1\end{center}
    }
}

\setlength{\droptitle}{-2em}

  \title{Ergebnisthemenblätter NOVANIMAL WPIII.1\\
Menü-Linien, ein veraltetes Konzept? Wie Gäste dazu bewegt werden
können vermehrt ressourcenleichte Gerichte zu wählen}
    \pretitle{\vspace{\droptitle}\centering\huge}
  \posttitle{\par}
    \author{Gian-Andrea Egeler, Priska Baur}
    \preauthor{\centering\large\emph}
  \postauthor{\par}
      \predate{\centering\large\emph}
  \postdate{\par}
    \date{Oktober 2018}


\begin{document}
\maketitle

\hypertarget{lead-teaser}{%
\subsubsection{1. Lead / Teaser}\label{lead-teaser}}

Die zentrale Bedeutung tierischer Produkte spiegelt sich im Angebot der
Gastronomie. Im Jahr 2015 sind Fleischgerichte (20.3\%) mit den
entsprechenden Beilagen (30.8\%) im Zusammenhang der
Ausser-Haus-Verpflegung die meistkonsumierten Speisen in der Schweiz
{[}@{]} (GastroSuisse, 2016). Gemäss der ersten in den Jahren 2014/2015
durchgeführten nationalen Ernährungserhebung menuCH essen 70 \% der
Bevölkerung zwischen 18 und 75 Jahren am Mittag auswärts (Bochud u.~a.
2017). Daher rückt die Gastronomie als zentraler Akteur einer
innovativen und nachhaltigen Ernährungswirtschaft ins Blickfeld. Welche
Innovationen in der Gastronomie könnten dazu beitragen, den
Pro-Kopf-Verbrauch an tierischen Nahrungsmitteln zu senken?

\hypertarget{forschungsfragen-und-ziele}{%
\subsubsection{2. Forschungsfragen und
Ziele}\label{forschungsfragen-und-ziele}}

Die übergeordnete Forschungsfrage dieses Arbeitspackets war es die
Gäste dazu zu bewegen, häufiger ressourcenschonende
vegetarische\footnote{vegetarisch = ovo-lakto-vegetarisch (inkl. Eier
  und Milch)} oder vegane\footnote{vegan = ausschliesslich pflanzliche
  Zustaten} Gerichte zu wählen. Die abgeleitete Forschungsfragen
interessiert es wie das Menü-Angebot oder die Menü-Beschriftung die
Menü-Wahl beeinflusst?

\hypertarget{theoretischer-hintergrund-und-methoden}{%
\subsubsection{3. Theoretischer Hintergrund und
Methoden}\label{theoretischer-hintergrund-und-methoden}}

Zusammen mit den Praxispartnern, einem Catering Unternehmen und dem
Facility Management einer Hochschule, wurde ein Quasi-Experiment
vorbereitet, das als Pilotversuch in zwei Hochschulmensen durchgeführt
wird. Konkret wird untersucht, wie die Gäste auf ein verändertes
Menü-Angebot mit einem höheren Anteil an ressourcenschonenden
vegetarischen oder veganen Gerichten reagieren. Das Quasi-Experiment
findet im Herbstsemester 2017 während 12 Wochen statt und besteht aus
zwei Mensazyklen Ã~ 6 Wochen (siehe Abbildung \ref{fig:fig1}). Ãœber den
gesamten Untersuchungszeitraum werden insgesamt 93 verschiedene Menüs
angeboten. In den 6 Referenz- bzw. Basiswochen wurden zwei fleisch- oder
fischhaltige Gerichte und ein vegetarisches Menü angeboten. In den 6
Interventionswochen wurde das Verhältnis umgekehrt und es wurden ein
veganes, ein vegetarisches und ein fleisch- oder fischhaltiges Gericht
angeboten. Beim veganen Angebot wurde zwischen eigenständige
authenische Gerichte (z. B. Linsen-Dal) und vegane Gerichte mit
Fleischsubstituten (z.B. Quornragout) unterschieden. Basis- und
Interventionsangebote wechselten wöchentlich ab. Während der gesamten
12 Wochen konnten die Gäste jeweils auf ein Buffet ausweichen und ihre
Mahlzeit aus warmen und kalten Komponenten selber zusammenstellen.
Versuchspläne, bei denen auf eine Interventionsphase wieder eine
Basisphase folgt, bezeichnet man als Umkehrpläne {[}@{]} (Jain \&
Spieß, 2012). Ziel dieser Umkehrdesigns ist es zu überprüfen, ob die
Interventionsphase, unter Kontrolle der Basisphase, einen Einfluss auf
das zu untersuchende Zielverhalten nimmt. In solchen speziellen
Versuchsplänen stellen die Teilnehmenden Einzelfälle dar. Diese
sogenannten single-subject Designs eignen sich gut, um
Ursache-Wirkung-Beziehungen zu eruieren {[}@{]} (Gravetter \& Forzano,
2016). Dieser ABABAB-Umkehrplan (oder Design) wurde gewählt, weil die
Mensabesucher per Definition Einzelfälle darstellen. In unserem
Quasi-Experiment dient dieses Design zu überprüfen, ob das veränderte
Menü-Angebot einen Einfluss auf das Wahlverhalten der Mensabesucher
nimmt. Die Abbildung \ref{fig:fig1} zeigt, wie die Gerichte über drei
vorgegebene Menü-Linien: Favorite, Kitchen, World währen des ersten
Mensazyklus randomisiert angeboten wurden. Ziel einer randomisierten
Ausgabe der Gerichte war es, einerseits die Mensagäste zu
«desorientieren» und andererseits Übungs- oder Gewöhnungseffekten zu
unterbinden. Übungs- oder Gewöhnungseffekten (engl. Testing effect or
habituation) können jedoch die interne Validität bedrohen. Eine
weitere Massnahme dem entgegenzuwirken war, dass wir im zweiten
Mensazyklus das Design zu einem BABABA-Design änderten.

\begin{figure}
\includegraphics[width=1\linewidth]{design_eng_experiment_ohne datum_180426_03egel} \caption{\label{fig:fig1} Die Abbildung 1 zeigt das Versuchsdesign der ersten 6 Experimentwochen (Kalenderwochen 40 bis 45).}\label{fig:unnamed-chunk-2}
\end{figure}

\hypertarget{ergebnisse}{%
\subsubsection{4. Ergebnisse}\label{ergebnisse}}

Wird ein grösseres, vielfältigeres und schmackhafteres das Angebot an
attraktiven vegetarischen und veganen Gerichten von den Mensabesuchern
häufiger gewählt? Von insgesamt 93 Menüs, welche pro Mensazyklus
angeboten wurden, enthielten 0 Prozent (n = 45) der Gerichte Fleisch.
Ein Drittel (n = 31) der Menüs wurde vegetarisch und die restlichen 0
Prozent (n = 16) ausschliesslich pflanzlich angeboten. Im Vergleich zu
den vergangenen Jahren (2015 und 2016) wuchs das vegetarische und vegane
Angebot im 2017 von 30 Prozent auf knapp 40 Prozent, wobei das Hot and
Cold-Buffet nicht als vegetarisches Angebot mitgezählt wurde. In der
Abbildung \ref{fig:fig2} zeigt alle verkauften Gerichte nach
Menü-Inhalt (vegan, vegetarisch, Fleisch und Buffet), welche während
des Quasi-Experiments angeboten wurden.

\begin{figure}
\centering
\includegraphics{ergebnisthemenblatt_i_181004_egel_files/figure-latex/unnamed-chunk-3-1.pdf}
\caption{\label{fig:fig2} Die Abbildung 2 zeigt die verkauften Gerichten
aufgeteilt nach den fünf Menü-Inhalten über die 12 Wochen}
\end{figure}

\includegraphics{ergebnisthemenblatt_i_181004_egel_files/figure-latex/unnamed-chunk-4-1.pdf}

\begin{verbatim}
## Dunn (1964) Kruskal-Wallis multiple comparison
\end{verbatim}

\begin{verbatim}
##   p-values adjusted with the Benjamini-Hochberg method.
\end{verbatim}

\begin{figure}
\centering
\includegraphics{ergebnisthemenblatt_i_181004_egel_files/figure-latex/unnamed-chunk-5-1.pdf}
\caption{\label{fig:fig3} Die Abbildung 3 zeigt die verkauften Gerichten
nach Menü-Inahlt und Bedingung (Intervention- und Basiswochen)}
\end{figure}

Der Abbildung \ref{fig:fig2} ist zu entnehmen, dass die wöchentlichen
Verkaufszahlen sich statistisch nicht unterscheiden (\emph{F}(1,10) =
284.209, \emph{p} = 0). Des Weiteren gab es Unterschiede in den
Verkaufszahlen zwischen den Interventions- und Basiswochen (\emph{H}(7)
= 42.56, \emph{p} \textless{} 0.001). Auffallend dabei sind, dass
deutlch weniger fleischhaltige Gerichte in den Basiswochen als in den
Interventionswochen konsumiert worden ist (siehe Abbildung
\ref{fig:fig3}).

\begin{figure}
\centering
\includegraphics{ergebnisthemenblatt_i_181004_egel_files/figure-latex/unnamed-chunk-6-1.pdf}
\caption{\label{fig:fig4} Die Abbildung 4 zeigt die verkauften Gerichten
nach Menü-Linien und Menü-Inhalt im Vergleich (2015, 2016 und 2017)}
\end{figure}

\hypertarget{diskussion-inkl.-einordnung-der-ergebnisse-in-die-internationale-forschung}{%
\subsubsection{5. Diskussion (inkl. Einordnung der Ergebnisse in die
internationale
Forschung)}\label{diskussion-inkl.-einordnung-der-ergebnisse-in-die-internationale-forschung}}

\hypertarget{schlussfolgerungen-mit-empfehlungenimplikationen-far-praxis-und-politik}{%
\subsubsection{6. Schlussfolgerungen mit Empfehlungen/Implikationen für
Praxis und
Politik}\label{schlussfolgerungen-mit-empfehlungenimplikationen-far-praxis-und-politik}}

\hypertarget{danksagung}{%
\subsubsection{7. Danksagung}\label{danksagung}}

Wir bedanken uns bei der SV Schweiz und insbesondere bei den konkreten
Verantwortlichen (Area- und Restaurantmanager und Küchenteam) für die
hilfsbereite und unkomplizierte Unterstützung (Lieferung von Daten,
Rezepturen, Beantwortung von Fragen).

\hypertarget{quellen}{%
\subsection*{8. Quellen}\label{quellen}}
\addcontentsline{toc}{subsection}{8. Quellen}

\hypertarget{refs}{}
\leavevmode\hypertarget{ref-bochud_anthropometric_2017}{}%
Bochud, Murielle, Angéline Chatelan, Juan-Manuel Blanco, und Sigrid
Beer-Borst. 2017. „Anthropometric characteristics and indicators of
eating and physical activity behaviors in the Swiss adult population``.
\url{https://www.blv.admin.ch/dam/blv/de/dokumente/lebensmittel-und-ernaehrung/ernaehrung/menuch-bericht.pdf.download.pdf/menuch-bericht.pdf}.


\end{document}
