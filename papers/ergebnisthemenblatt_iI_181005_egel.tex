\documentclass[12pt,ngerman,]{article}
\usepackage{lmodern}
\usepackage{amssymb,amsmath}
\usepackage{ifxetex,ifluatex}
\usepackage{fixltx2e} % provides \textsubscript
\ifnum 0\ifxetex 1\fi\ifluatex 1\fi=0 % if pdftex
  \usepackage[T1]{fontenc}
  \usepackage[utf8]{inputenc}
\else % if luatex or xelatex
  \ifxetex
    \usepackage{mathspec}
  \else
    \usepackage{fontspec}
  \fi
  \defaultfontfeatures{Ligatures=TeX,Scale=MatchLowercase}
\fi
% use upquote if available, for straight quotes in verbatim environments
\IfFileExists{upquote.sty}{\usepackage{upquote}}{}
% use microtype if available
\IfFileExists{microtype.sty}{%
\usepackage{microtype}
\UseMicrotypeSet[protrusion]{basicmath} % disable protrusion for tt fonts
}{}
\usepackage[margin = 1.2in]{geometry}
\usepackage{hyperref}
\hypersetup{unicode=true,
            pdfauthor={Gian-Andrea Egeler, Priska Baur},
            pdfborder={0 0 0},
            breaklinks=true}
\urlstyle{same}  % don't use monospace font for urls
\ifnum 0\ifxetex 1\fi\ifluatex 1\fi=0 % if pdftex
  \usepackage[shorthands=off,main=ngerman]{babel}
\else
  \usepackage{polyglossia}
  \setmainlanguage[]{german}
\fi
\usepackage{graphicx,grffile}
\makeatletter
\def\maxwidth{\ifdim\Gin@nat@width>\linewidth\linewidth\else\Gin@nat@width\fi}
\def\maxheight{\ifdim\Gin@nat@height>\textheight\textheight\else\Gin@nat@height\fi}
\makeatother
% Scale images if necessary, so that they will not overflow the page
% margins by default, and it is still possible to overwrite the defaults
% using explicit options in \includegraphics[width, height, ...]{}
\setkeys{Gin}{width=\maxwidth,height=\maxheight,keepaspectratio}
\IfFileExists{parskip.sty}{%
\usepackage{parskip}
}{% else
\setlength{\parindent}{0pt}
\setlength{\parskip}{6pt plus 2pt minus 1pt}
}
\setlength{\emergencystretch}{3em}  % prevent overfull lines
\providecommand{\tightlist}{%
  \setlength{\itemsep}{0pt}\setlength{\parskip}{0pt}}
\setcounter{secnumdepth}{0}
% Redefines (sub)paragraphs to behave more like sections
\ifx\paragraph\undefined\else
\let\oldparagraph\paragraph
\renewcommand{\paragraph}[1]{\oldparagraph{#1}\mbox{}}
\fi
\ifx\subparagraph\undefined\else
\let\oldsubparagraph\subparagraph
\renewcommand{\subparagraph}[1]{\oldsubparagraph{#1}\mbox{}}
\fi

%%% Use protect on footnotes to avoid problems with footnotes in titles
\let\rmarkdownfootnote\footnote%
\def\footnote{\protect\rmarkdownfootnote}

%%% Change title format to be more compact
\usepackage{titling}

% Create subtitle command for use in maketitle
\newcommand{\subtitle}[1]{
  \posttitle{
    \begin{center}\large#1\end{center}
    }
}

\setlength{\droptitle}{-2em}

  \title{Ergebnisthemenblätter NOVANIMAL\\
Wer wählt welches Menü? Ernährungsmuster in einer Hochschulmensa}
    \pretitle{\vspace{\droptitle}\centering\huge}
  \posttitle{\par}
    \author{Gian-Andrea Egeler, Priska Baur}
    \preauthor{\centering\large\emph}
  \postauthor{\par}
      \predate{\centering\large\emph}
  \postdate{\par}
    \date{Oktober 2018}

\usepackage{fancyhdr}
\usepackage{placeins}
\usepackage{setspace}
\usepackage{chngcntr}
\onehalfspacing

\begin{document}
\maketitle

\#bibliography: biblio\_181004.bib

\hypertarget{lead-teaser}{%
\subsubsection{1. Lead / Teaser}\label{lead-teaser}}

Täglich fällen Menschen in all ihren Lebensbereichen unzählige
Entscheidungen. Konsumentscheidungen unterscheiden sich - manche können
schnell und leicht getroffen werden, andere sind komplex und erfordern
ausführliche Verarbeitung von Informationen. In der Literatur wird die
Konsumentscheidung oftmals als einen Prozess der Kaufentscheidung
beschrieben {[}@{]} (Lamb, Hair, \& McDaniel, 2012). Dieser Kauf wird
von kognitiver und emotionaler Einflüssen, Familie, Freunde, Werbungen,
Vorbilder, Stimmungen und Situation aber auch von unterschiedlichen
Generationen (oder Alter) beeinflusst {[}@{]} (Schiffman, Kanuk, \&
Wisen-blit, 2010; Wai San Yap \& Yazdanifard, 2014). Welche Innovationen
in der Gastronomie könnten dazu beitragen, die Konsumentschaiedung
nachhaltiger zu gestalten?

\hypertarget{forschungsfragen-und-ziele}{%
\subsubsection{2. Forschungsfragen und
Ziele}\label{forschungsfragen-und-ziele}}

Die übergeordnete Forschungsfrage dieses Arbeitspackets WPIII.1 war es
die Gäste dazu zu bewegen, häufiger ressourcenschonende
vegetarische\footnote{vegetarisch = ovo-lakto-vegetarisch (inkl. Eier
  und Milch)} oder vegane\footnote{vegan = ausschliesslich pflanzliche
  Zustaten} Gerichte zu wählen. Die abgeleitete Forschungsfragen
interessiert es, wie das Menü-Angebot die Menü-Wahl der verschiedener
Gästegruppen beeinflusst? Gibt es spezifische Ernährungsmuster in
Hochschulmensen?

\hypertarget{theoretischer-hintergrund-und-methoden}{%
\subsubsection{3. Theoretischer Hintergrund und
Methoden}\label{theoretischer-hintergrund-und-methoden}}

Menschen entscheiden begrenzt rational, weil sie über begrenzte
Informationen, Intelligenz und Zeit verfügen {[}@{]} (Simon, 1956).
Zudem vermeiden Menschen komplexe Entscheidungen und bleiben oftmals
beim Status Quo. Menschen können beispielsweise Risiken schlecht
abschätzen oder es fällt ihnen schwer, kurzfristigen Nutzen für
langfristigen Nutzen aufzugeben. Dadurch realisieren Menschen bei ihren
Entscheidungen oft nicht immer den grösstmöglichen Nutzen für sich
selbst {[}@{]} (Felser, 2015). Habituelle Konsumentscheidungen basieren
auf der Annahme, dass irrationale Entscheide nicht wesentlich schlechter
als aufwändigere, rationale Kaufentscheidungen sind. Choice Architecture
und Heuristiken gehören zu diesen habituellen Konsumentscheidungen.
Konsumentscheidungen können auf verschiedene Wege beeinflusst werden.
Einerseits kann eine Einstellungsänderung dazu führen, dass der
Entscheider sein Verhalten verändert. Andererseits kann durch eine
Gestaltung der kontextuellen Hinweisreize ein Verhalten geändert werden.
Mit dem Letzteren, eine Änderung der Entscheidungsumgebungen, befasst
sich die Choice Architecture (Thaler \& Sunstein, 2003, 2008). Choice
Architecture nutzt Tools, um Entscheidungen zu beeinflussen. In unserer
Studie wurde die Entscheidungsumgebung verändert. In den 6 Referenz-
bzw. Basiswochen wurden zwei fleisch- oder fischhaltige Gerichte und ein
vegetarisches Gericht angeboten. In den 6 Interventionswochen wurde das
Verhältnis umgekehrt und es wurden ein veganes, ein vegetarisches und
ein fleisch- oder fischhaltiges Gericht angeboten. Das Ziel war es zu
schauen, wie dieses veränderte Menü-Angebot die Menü-Wahl der
Mensabesucher verändert. Versuchspläne, bei denen auf eine
Interventionsphase wieder eine Basisphase folgt, bezeichnet man als
Umkehrpläne {[}@jain\_versuchsplane\_2012{]}. Ziel dieser Umkehrdesigns
ist es zu überprüfen, ob die Interventionsphase, unter Kontrolle der
Basisphase, einen Einfluss auf das zu untersuchende Zielverhalten nimmt.
In solchen speziellen Versuchsplänen stellen die Teilnehmenden
Einzelfälle dar. Diese sogenannten single-subject Designs eignen sich
gut, um Ursache-Wirkung-Beziehungen zu eruieren
{[}@gravetter\_research\_2016{]}. Dieser ABABAB-Umkehrplan (oder Design)
wurde gewählt, weil die Mensabesucher per Definition Einzelfälle
darstellen. In unserem Quasi-Experiment dient dieses Design zu
überprüfen, ob das veränderte Menü-Angebot einen Einfluss auf das
Wahlverhalten der Mensabesucher nimmt. Die Gerichte wurden über drei
vorgegebene Menü-Linien: Favorite, Kitchen, World während des ersten
Mensazyklus randomisiert angeboten wurden. Ziel einer randomisierten
Ausgabe der Gerichte war es, einerseits die Mensagäste zu
«desorientieren» und andererseits Übungs- oder Gewöhnungseffekten zu
unterbinden. Übungs- oder Gewöhnungseffekten (engl. Testing effect or
habituation) können jedoch die interne Validität bedrohen. Eine weitere
Massnahme dem entgegenzuwirken war, dass wir im zweiten Mensazyklus das
Design zu einem BABABA-Design änderten. Die nachfolgenden Analysen und
Ergebnisse basieren auf zusammengeführte Daten aus dem Kassensysten des
Catering Unternehmens und dem System der Hochschule. Diese Verkaufsdaten
stellen somit einen Proxy für das Wahlverhalten(oder so / oder ist mehr
als nur eine Intention für ein Wahlverhalten, sondern eher ein Proxy für
das tatsächliche Wahlverhalten, evtl Literatur). Für die Analysen sind
insgesamt mehr als 22'000 Transaktionen resp. Käufe von Gerichten
während der Mittagszeit berücksichtigt worden.

\hypertarget{ergebnisse}{%
\subsubsection{4. Ergebnisse}\label{ergebnisse}}

\hypertarget{diskussion-inkl.-einordnung-der-ergebnisse-in-die-internationale-forschung}{%
\subsubsection{5. Diskussion (inkl. Einordnung der Ergebnisse in die
internationale
Forschung)}\label{diskussion-inkl.-einordnung-der-ergebnisse-in-die-internationale-forschung}}

price studie evtl reinnehmen siehe buch von lynn fewer 2007

choice sets erhöhen!! thaler\_nudge:\_2008 =\textgreater{} per
definition ein nudge. Der entscheidende Einfluss der Erweiterung des
Wahlsets unterstreicht die Idee, die Verbraucher durch einfache
Veränderungen der Umwelt zu günstigen Entscheidungen anzuregen.

\hypertarget{schlussfolgerungen-mit-empfehlungenimplikationen-fur-praxis-und-politik}{%
\subsubsection{6. Schlussfolgerungen mit Empfehlungen/Implikationen für
Praxis und
Politik}\label{schlussfolgerungen-mit-empfehlungenimplikationen-fur-praxis-und-politik}}

Gründe für Menü-Wahl sind heterogen (siehe bakker/dagevous 2012) Fokus
auf Flexitarier (siehe Bakker und Dagevous 2012) Fokus auf Frauen (siehe
Ruby und heine 2011)

Dank der CampusCard, einer Smard-Card Technologie konnten die
Ernährungsweisen einzelner Personen ohne den Datenschutz zu verletzten,
aufgezeichnet werden. Bei einer Änderung des Angebots muss zuerst die
Besucher eruiert werden und was die essen, etc {[}siehe
@lambert\_using\_2005{]}.

\hypertarget{danksagung}{%
\subsubsection{7. Danksagung}\label{danksagung}}

Wir bedanken uns bei der SV Schweiz und insbesondere bei den konkreten
Verantwortlichen (Area- und Restaurantmanager und Küchenteam) für die
hilfsbereite und unkomplizierte Unterstützung (Lieferung von Daten,
Rezepturen, Beantwortung von Fragen).

\hypertarget{quellen}{%
\subsubsection{8. Quellen}\label{quellen}}


\end{document}
