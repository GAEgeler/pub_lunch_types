\documentclass[12pt,ngerman,]{article}
\usepackage{lmodern}
\usepackage{amssymb,amsmath}
\usepackage{ifxetex,ifluatex}
\usepackage{fixltx2e} % provides \textsubscript
\ifnum 0\ifxetex 1\fi\ifluatex 1\fi=0 % if pdftex
  \usepackage[T1]{fontenc}
  \usepackage[utf8]{inputenc}
\else % if luatex or xelatex
  \ifxetex
    \usepackage{mathspec}
  \else
    \usepackage{fontspec}
  \fi
  \defaultfontfeatures{Ligatures=TeX,Scale=MatchLowercase}
\fi
% use upquote if available, for straight quotes in verbatim environments
\IfFileExists{upquote.sty}{\usepackage{upquote}}{}
% use microtype if available
\IfFileExists{microtype.sty}{%
\usepackage{microtype}
\UseMicrotypeSet[protrusion]{basicmath} % disable protrusion for tt fonts
}{}
\usepackage[margin = 1.2in]{geometry}
\usepackage{hyperref}
\hypersetup{unicode=true,
            pdfauthor={Gian-Andrea Egeler, Priska Baur},
            pdfborder={0 0 0},
            breaklinks=true}
\urlstyle{same}  % don't use monospace font for urls
\ifnum 0\ifxetex 1\fi\ifluatex 1\fi=0 % if pdftex
  \usepackage[shorthands=off,main=ngerman]{babel}
\else
  \usepackage{polyglossia}
  \setmainlanguage[]{german}
\fi
\usepackage{graphicx,grffile}
\makeatletter
\def\maxwidth{\ifdim\Gin@nat@width>\linewidth\linewidth\else\Gin@nat@width\fi}
\def\maxheight{\ifdim\Gin@nat@height>\textheight\textheight\else\Gin@nat@height\fi}
\makeatother
% Scale images if necessary, so that they will not overflow the page
% margins by default, and it is still possible to overwrite the defaults
% using explicit options in \includegraphics[width, height, ...]{}
\setkeys{Gin}{width=\maxwidth,height=\maxheight,keepaspectratio}
\IfFileExists{parskip.sty}{%
\usepackage{parskip}
}{% else
\setlength{\parindent}{0pt}
\setlength{\parskip}{6pt plus 2pt minus 1pt}
}
\setlength{\emergencystretch}{3em}  % prevent overfull lines
\providecommand{\tightlist}{%
  \setlength{\itemsep}{0pt}\setlength{\parskip}{0pt}}
\setcounter{secnumdepth}{0}
% Redefines (sub)paragraphs to behave more like sections
\ifx\paragraph\undefined\else
\let\oldparagraph\paragraph
\renewcommand{\paragraph}[1]{\oldparagraph{#1}\mbox{}}
\fi
\ifx\subparagraph\undefined\else
\let\oldsubparagraph\subparagraph
\renewcommand{\subparagraph}[1]{\oldsubparagraph{#1}\mbox{}}
\fi

%%% Use protect on footnotes to avoid problems with footnotes in titles
\let\rmarkdownfootnote\footnote%
\def\footnote{\protect\rmarkdownfootnote}

%%% Change title format to be more compact
\usepackage{titling}

% Create subtitle command for use in maketitle
\newcommand{\subtitle}[1]{
  \posttitle{
    \begin{center}\large#1\end{center}
    }
}

\setlength{\droptitle}{-2em}

  \title{Kurzberichte NOVANIMAL\\
Transdisziplinäres Experiment in zwei Hochschulmensen: Auswertungen der
Menü-Verkäufe}
    \pretitle{\vspace{\droptitle}\centering\huge}
  \posttitle{\par}
    \author{Gian-Andrea Egeler, Priska Baur}
    \preauthor{\centering\large\emph}
  \postauthor{\par}
      \predate{\centering\large\emph}
  \postdate{\par}
    \date{November 2018}

\usepackage{fancyhdr}
\usepackage{setspace}
\onehalfspacing
\usepackage{float}
\usepackage{graphicx}
\usepackage[font=small,labelfont=bf]{caption}
\usepackage{apacite}
\bibliographystyle{apacite}
\bibliography{biblio_181004.bib}

\begin{document}
\maketitle

\hypertarget{lead-teaser}{%
\subsubsection{1. Lead / Teaser}\label{lead-teaser}}

Die zentrale Bedeutung tierischer Produkte spiegelt sich im Angebot der
Gastronomie. Im Jahr 2015 sind Fleischgerichte (20.3\%) mit den
entsprechenden Beilagen (30.8\%) im Zusammenhang der
Ausser-Haus-Verpflegung die meistkonsumierten Speisen in der Schweiz
{[}@gastrosuisse\_branchenspiegel\_2016{]}. Gemäss der ersten in den
Jahren 2014/2015 durchgeführten nationalen Ernährungserhebung menuCH
essen 70 \% der Bevölkerung zwischen 18 und 75 Jahren am Mittag auswärts
{[}@bochud\_anthropometric\_2017{]}. Daher rückt die
Gemeinschaftsgastronomie als zentraler Akteur einer innovativen und
nachhaltigen Ernährungswirtschaft ins Blickfeld. Welche Innovationen in
der Gemeinschaftsgastronomie könnten dazu beitragen, den
Pro-Kopf-Verbrauch an tierischen Nahrungsmitteln zu senken?

\hypertarget{forschungsfragen-und-ziele}{%
\subsubsection{2. Forschungsfragen und
Ziele}\label{forschungsfragen-und-ziele}}

Die übergeordnete Forschungsfrage dieses Arbeitspackets WPIII.1 war es,
die Gäste in Mensen dazu zu bewegen, häufiger ressourcenschonende
vegetarische\footnote{vegetarisch = ovo-lakto-vegetarisch (inkl. Eier
  und Milch)} oder vegane\footnote{vegan = ausschliesslich pflanzliche
  Zustaten} Gerichte zu wählen. Die daraus abgeleiteten Forschungsfragen
sind wie das Menü-Angebot und die Menü-Beschriftung die Menü-Wahl der
Mensabesucher beeinflusst?

\newpage

\hypertarget{theoretischer-hintergrund-und-methoden}{%
\subsubsection{3. Theoretischer Hintergrund und
Methoden}\label{theoretischer-hintergrund-und-methoden}}

Zusammen mit den Praxispartnern, einem Catering Unternehmen und dem
Facility Management einer Hochschule, wurde ein Experiment vorbereitet,
das als Pilotversuch in zwei Hochschulmensen durchgeführt wird. Konkret
wird untersucht, wie die Gäste auf ein verändertes Menü-Angebot mit
einem höheren Anteil an ressourcenschonenden vegetarischen oder veganen
Gerichten reagieren.

\par

Das Experiment findet im Herbstsemester (HS) 2017 während 12 Wochen
statt und bestand aus zwei Mensazyklen à 6 Wochen (siehe Abbildung
\ref{fig:fig1}). In den 6 Referenz- bzw. Basiswochen wurden zwei
fleisch- oder fischhaltige Gerichte und ein vegetarisches Gericht
angeboten. In den 6 Interventionswochen wurde das Verhältnis umgekehrt
und es wurden ein veganes, ein vegetarisches und ein fleisch- oder
fischhaltiges Gericht angeboten. Beim veganen Angebot wurde zwischen
eigenständige authenische Gerichte\footnote{Gerichte, welche
  ausschliesslich aus pflanzlichen Zutaten bestehen z. B. Linsen-Dal}
und vegane Gerichte mit Fleischsubstituten\footnote{Gerichte, welche
  neben pflanzlichen Zutaten auch noch Fleischsubstitute beinhalteten z.
  B. Tofu-Burger} unterschieden. Basis- und Interventionsangebote
wechselten wöchentlich ab. Während der gesamten 12 Wochen konnten die
Gäste jeweils auf ein Buffet ausweichen und ihre Mahlzeit aus warmen und
kalten Komponenten selber zusammenstellen.

\par

Die nachfolgenden Analysen und Ergebnisse basieren auf Daten aus den
Kassensystemen des Catering Unternehmens. Für die Analysen sind mehr als
26'000 Menü-Verkäufe resp. Transaktionen während der Mittagszeit
berücksichtigt worden. Alle Analysen wurden mit dem Statistikprogramm R
(R version 3.4.3 (2017-11-30)) gerechnet.

\begin{figure}[H]

{\centering \includegraphics[width=1\linewidth]{design_eng_experiment_ohne datum_180426_03egel} 

}

\caption{\label{fig:fig1} Die Abbildung zeigt das Versuchsdesign der ersten 6 Experimentwochen (Kalenderwochen 40 bis 45).}\label{fig:unnamed-chunk-1}
\end{figure}

\hypertarget{ergebnisse}{%
\subsubsection{4. Ergebnisse}\label{ergebnisse}}

\hypertarget{verkaufszahlen-uber-alle-12-wochen}{%
\paragraph{4.1 Verkaufszahlen über alle 12
Wochen}\label{verkaufszahlen-uber-alle-12-wochen}}

Im Vergleich zu den vergangenen HS 2015 und 2016 wuchs das vegetarische
und vegane Angebot im HS 2017 von 40 Prozent auf knapp 47 Prozent, wobei
das Hot and Cold-Buffet nicht als vegetarisches Angebot mitgezählt
wurde.

\par

Die totalen Verkaufszahlen des HS 2017 haben sich im Vergleich zu den
zwei HS 2015 und 2016 nur marginal verändert. Die Abbildung
\ref{fig:fig2} zeigt alle verkauften Gerichten nach Menü-Inhalt, welche
während des Experiments verkauft wurden.

\begin{figure}[H]

{\centering \includegraphics[width=1\linewidth]{ergebnisthemenblatt_i_181004_egel_files/figure-latex/unnamed-chunk-2-1} 

}

\caption{\label{fig:fig2} Die Abbildung zeigt alle verkauften Gerichten (\textit{N} = 26'234) aufgeteilt nach den fünf Menü-Inhalten über die 12 Wochen.}\label{fig:unnamed-chunk-2}
\end{figure}

Der Abbildung \ref{fig:fig2} ist zu entnehmen, dass die wöchentlichen
Verkaufszahlen im HS 2017 sich statistisch nicht unterscheiden
(\emph{F}(1,10) = 0.121, \emph{p} \textless{} .001). Des Weiteren gab es
Unterschiede in den Verkaufszahlen zwischen den Basis- und
Interventionswochen (\emph{F}(7, 39) = 594.63, \emph{p} \textless{}
.001). Neben den schwankenden verkaufszahlen der Fleischgerichte, fällt
es auf, dass vegane authentische Gerichte deutlich besser verkauft
wurden als vegan fleischsubstitituierte Gerichte.

\hypertarget{menu-optionen}{%
\paragraph{4.2 Menü-Optionen}\label{menu-optionen}}

Von insgesamt 93 verschiedenen Gerichten, welche über zwei
Mensazyklen\footnote{ein Zyklus besteht aus 6 Wochen, danach wiederholen
  sich die Gerichten} verteilt angeboten wurden, enthielten 48 Prozent
(\emph{n} = 45) der Gerichte Fleisch. Ein Drittel (\emph{n} = 31) der
Gerichte wurde vegetarisch und die restlichen 17 Prozent (\emph{n} = 16)
ausschliesslich pflanzlich angeboten.

\par

Neunzig verschiedenen Mahlzeiten wurden pro Mensa und Mensazyklus auf
drei Menü-Linien (Favorite, Kitchen und World) angeboten. Wird das
gesamte Angebot auf beide Mensen und über alle Tage addiert, ergibt das
ein geplantes Angebot von 360 Gerichten\footnote{90 Gerichte x 2
  Mensazyklen x 2 Mensen}. Aus betrieblichen Gründen kam es beim
Catering Unternehmen des Öfteren zu Zusatzangeboten, welche das geplante
Angebot zusätzlich vergrösserte. In den 12 Wochen wurden zu den total
geplanten 360 Gerichten noch 120 Gerichten zusätzlich angeboten (siehe
Abbildung \ref{fig:fig3}).

\begin{figure}[H]

{\centering \includegraphics[width=1\linewidth]{ergebnisthemenblatt_i_181004_egel_files/figure-latex/unnamed-chunk-5-1} 

}

\caption{\label{fig:fig3} Die Abbildung zeigt das geplante (\textit{n} = 480) und angebotene Angebot (\textit{n} = 600) für ein ganzes Herbstsemester (60 Tage) und zwei Hochschulmensen nach den fünf Menü-Inhalten.}\label{fig:unnamed-chunk-5}
\end{figure}

Die Abbildung \ref{fig:fig3} zeigt für die Herbstemester die kalkulierte
Planung von 2015 und 2016 als auch das geplante Angebot des Experiments
im HS 2017. Die dritte Säule zeigt die tatsächlich angebotene Auswahl
für das HS 2017. Das geplante Angebot und das tatsächliche Angebot im HS
2017 unterscheiden sich in den Menü-Inhalten\footnote{Fleisch oder
  Fisch, Vegetarisch, Vegan (authentisch) und Vegan (Fleischsubstitut)}
nicht statisch (\(\chi^2\)(3) = 1.59, \emph{p} = .662). Auch beim
näheren Betrachten unterschieden sich die geplanten und tatsächlichen
Angeboten für das HS 2017 in den Menü-Inhalten zwischen den Basis-
(\(\chi^2\)(3) = 7.14, \emph{p} = .067) und Interventionswochen
(\(\chi^2\)(3) = 5.98, \emph{p} = .112) nicht.

\par

Wird das tatsächliche Angebot näher beleuchtet, wurden in den
Interventionenwochen (\emph{n} = 238) 40 Prozent fleischhaltige, knapp
einem Drittel vegetarische und 27 Prozent vegane Gerichte angeboten. In
den 12 Basiswochen (\emph{n} = 242) entiehlten \(\frac{2}{3}\) Prozent
des Angebots Fleisch oder Fisch. Vegetarische Gerichte machten rund 29
Prozent und vegane Gerichte knapp ein Prozent aus.

\hypertarget{einfluss-menu-angebot}{%
\paragraph{4.3 Einfluss Menü-Angebot}\label{einfluss-menu-angebot}}

Wird ein grösseres und vielfältigeres Angebot an attraktiven
vegetarischen und veganen Gerichten von den Mensabesuchern häufiger
gewählt? Es zeigten sich Unterschiede in den Verkaufszahlen zwischen den
Versuchsbedingungen un den Menü-Inhalten. Post-hoc-Analysen (nach Tukey)
zeigten, dass es keine statistischen Unterschiede in den Verkaufszahlen
beim Hot and Cold und im vegetarischen Angebot zwischen den beiden
Versuchsbedingungen ``Basis'' und ``Intervention'' gab. Auffallend sind
die signifikanten Unterschiede im Verkauf von Fleischgerichten, welche
deutlich seltener in den Interventionswochen als in den Basiswochen
verkauft wurden. Dass sich mehr vegane Gerichte in den
Interventionswochen gekauft wurden lag daran, dass es in den Basiswochen
lediglich ein kleines veganes Angebot gab (siehe Abbildung
\ref{fig:fig4}).

\begin{figure}[H]

{\centering \includegraphics[width=1\linewidth]{ergebnisthemenblatt_i_181004_egel_files/figure-latex/unnamed-chunk-7-1} 

}

\caption{\label{fig:fig4} Die Abbildung zeigt die verkauften Gerichten nach Menü-Inhalt und Bedingung (Intervention- und Basiswochen).}\label{fig:unnamed-chunk-7}
\end{figure}

\hypertarget{einfluss-menu-linien}{%
\paragraph{4.4 Einfluss Menü-Linien}\label{einfluss-menu-linien}}

Sind die aktuellen Menü-Linien tatsächlich ein veraltetes Konzept? Nebst
der Änderung im Angebot, war eine weitere Intervention die Gerichte über
neutrale Menü-Gefässe sog. Menü-Linien zu vertreiben. Für das Experiment
wurde die Menü-Linie Green zu World unbenannt, damit die Menü-Inhalte
randomisiert über diese drei Menü-Linien angeboten werden konnten. Die
Abbildung \ref{fig:fig5} vergleicht die Verkaufszahlen gemessen an den
Menü-Linien der Jahren 2015, 2016 und 2017.

\begin{figure}[H]

{\centering \includegraphics{ergebnisthemenblatt_i_181004_egel_files/figure-latex/unnamed-chunk-8-1} 

}

\caption{\label{fig:fig5} Die Abbildung zeigt die verkauften Gerichten nach Menü-Linien im Vergleich (2015, 2016 und 2017).}\label{fig:unnamed-chunk-8}
\end{figure}

Die Abbildung \ref {fig:fig5} zeigt, dass es zwischen den drei
Herbstsemestern klare Unterschiede in den Verkauszahlen zwischen den
Menü-Linien gab. Die Grafik zwar lässt keine Schlüsse über deren
Menü-Inhalt zu, aber die randomisierte Ausgabe der Gerichte konnte
aufzeigen, dass es einen Einfluss auf die Menü-Wahl genommen hat. So
wurden beispielsweise auch vegetarischen Gerichte auf der teureren
Menü-Linie «Kitchen» problemlos von den Gästen gekauft.

\newpage

\hypertarget{diskussion}{%
\subsubsection{5. Diskussion}\label{diskussion}}

Nach unserem Wissen wurde noch kein Real-Labor-Experiment durchgeführt,
welches die Reaktion der Gäste auf ein ressourcenschonenderes Angebot
aufzeichnete. Es gibt einige Untersuchungen wobei mittels Experimenten
getestet wurde wie die Mensagäste auf bestimmte Labels (z. B. zu
gesunder Ernährung oder umweltfreundlichen Gerichten)
{[}@hoefkens\_what\_2012; @spaargaren\_consumer\_2013;
@thorndike\_2-phase\_2012{]}, auf Präsentationen oder Positionierung von
Nahrungsmitteln {[}@wansink\_slim\_2013; @van\_kleef\_healthy\_2012;
@Bucher el at.{]} oder auf soziale oder injuktive Normen
{[}@collins\_two\_2019{]} reagierten.

\par

Zusammengefasst zeigen die Ergebnisse dieser Studie, dass eine Erhöhung
des vegetarischen und veganen Angebots in Kombination mit neutralen
Menü-Lienien und einer randomisierten Ausgabe des Menü-Inahlts zu einer
Senkung des Fleischkonsums führen könnten.\\
Eine mögliche Erklärung für die Senkung des Fleischkonsums ist die
Erhöhung der Verfügbarkeit der vegetarischen und veganen Gerichten.
Gemäss früheren Studien {[}siehe @blanchette\_determinants\_2005;
@neumark-sztainer\_factors\_1999; @van\_kleef\_exploiting\_2015{]} führt
eine höhere Verfügbarkeit, z. B. von gesundem Essen, auch zu einem
Einfluss auf das Ernährungsverhalten. Ensaff und Kollegen
{[}-@ensaff\_food\_2015{]} konnten in ihrer Studie aufzeigen, dass die
Verfügbarkeit als Nudging-Strategie eine Veränderung im Kaufverhalten
von pflanzenbasierten Nahrungsmitteln hervorgerufen hat. Ein weiterer
Einflussfaktor könnte das vergrösserte Angebot, welches durch das
Caternig Unternehmen generiert wurde, darstellen. Denn Aschemann-Witzel
und Kollegen {[}-@aschemann-witzel\_effects\_2013{]} fanden in ihrer
Studie heraus, dass eine Erhöhung der Auswahlmenge einen Einfluss auf
das gesunde Wahlverhalten nimmt.

\par

In den Interventionswochen liessen sich die veganen authentische
Gerichte besser verkauften als die veganen Gerichte mit
Fleischsubstituten. Dieses Ergebnis unserer Studie geht im Einklang mit
Ergebnis früherer Studien, welche zeigten, dass die Akzeptanz von
Fleischsubstituten immernoch relativ klein ist {[}siehe
@hoek\_will\_2010; @de\_boer\_merits\_2011{]}. Zudem fanden Elzerman und
Kollegen {[}-@elzerman\_consumer\_2011{]} heraus, dass der
Gerichte-Kontext (z. B. Ingredienzen) einen starken Einfluss auf den
Konsum von Fleischsubstituten nimmt.

\par

Obwohl die Menü-Linien resp. Menü-Labels oftmals für die Gästeführung
eingeführt werden, konnte im Experiment aufgezeigt werden, dass durch
die Wahl von neutralen Menü-Linien und eine randomisierte Ausgabe der
Gerichte auch teurere vegetarische und vegane Gerichte von den
Mensabesucher gekauft werden. Auch in der Literatur wird kontrovers
disskutiert, ob Labels überhaupt einen Einfluss auf die Menü-Wahl nehmen
{[}siehe @spaargaren\_consumer\_2013; @kiszko\_influence\_2014;
@aschemann-witzel\_effects\_2013{]}.

\par

Ob nun die randomisierte Ausgabe, die neutralen Menü-Linien, die
quantitative Erhöhung von vegetarische und vegane Gerichten oder die
Kombination dieser Interventionen für die Senkung des Fleischkonsums
zuständig ist, bedarf es an weiteren Untersuchungen. Weiter konnte in
diesem Experiment die qualitative Gleichwertigkeit der täglich
angebotenen Gerichte nicht überprüft werden, was gegebenenfalls bei dem
Kauf eines Gerichts einen Einfluss nehmen könnte {[}siehe @myung
2008{]}. Zudem basieren die gezeigten Ergebnisse lediglich auf
aggregierte Verkaufszahlen, was keine Schlüsse auf kausale Zusammenhänge
zulassen.

\par

Eine Kaufentscheidung zu verstehen und demnach zu beeinflussen ist sehr
komplex, denn wir fällen über 200 Essensentscheide pro Tag
{[}@wansink\_mindless\_2007{]}. Es zeigt sich aber immer mehr, dass die
meisten Entscheide anhand einfachen Heuristiken sog. Faustregeln gefällt
werden {[}siehe @scheibehenne\_fast\_2007{]}. Inwiefern die Intervention
auch tatsächlich einen Einfluss auf das individuelle Wahlverhalten
nimmt, wird in einem separaten Kurzbericht thematisiert (siehe Egeler \&
Baur, 2018b).

\hypertarget{schlussfolgerungen}{%
\subsubsection{6. Schlussfolgerungen}\label{schlussfolgerungen}}

Eine Erhöhung von Anzahl und Anteil an vegetarischen und veganen
Gerichten in Kombination mit neutralen Menü-Linien und randomisierter
Ausgabe der drei Menü-Inhalten (Fleisch oder Fisch, vegetarisch und
vegan), scheint den Konsum von Fleischgerichten zu reduzieren. An diesem
Punkt ist es wichtig zu erwähnen, dass die vegetarischen und veganen
Gerichte diskret neben anderen Inhaltsstoffen deklariert und nicht
angepriesen wurden. Ziel war es mit diesen Interventionen den Gast
vermehrt dazu führen, sich mit dem Menü-Inhalt auseinanderzusetzten,
alte Gewohnheiten fallen zu lassen und die Neophobie zu verkleinern.

\par

Wenn es um den einzelnen Verkauf von veganen Gerichten geht, scheinen
authentische Gerichte besser als Gerichte mit Fleischsubstitute
abzuschneiden. Daher nicht nur Fleischsubstitute anbieten und wenn
Fleischsubstitute angeboten werden, sollte auf den Menü-Kontext geachten
werden. Zudem zeigte sich, dass auch vegetarische und vegane Gerichte
auf der teureren Menü-Linie ``Kitchen'' ebenfalls von den Gästen
akzeptiert und gekauft wurden.

\par

In diesem Experiment konnte die qualitative Gleichwertigkeit der
angebotenen Gerichte nicht berücksichtigt werden. Bei einem weiteren
Mensa-Experiment sollte die Gleichwertigkeit der Gerichte bei der
Angebotsentwicklung mittels Fachleuten oder Koch-Experten berücksichtigt
werden.

\par

Mit diesem Experiment konnte gezeigt werden, dass sich die
Verkaufszahlen über die Zeit durch eine Erhöhung des vegetarischen und
veganen Angebots nicht zurückgegangen sind. Das spezielle Setting und
die spezielle Stichprobe haben bestimmt dafür beigetragen,
nichtsdestotrotz wären weitere Pilotstudien an anderen Kantinen
wünschenswert.

Weitere Informationen zum Projekt können auf unserer
\href{novanimal.ch}{Webpage} gefunden werden.

\hypertarget{danksagung}{%
\subsubsection{7. Danksagung}\label{danksagung}}

Wir bedanken uns bei der SV Schweiz und insbesondere bei den konkreten
Verantwortlichen (Area- und Restaurantmanager und Küchenteam) für die
hilfsbereite und unkomplizierte Unterstützung (Lieferung von Daten,
Rezepturen, Beantwortung von Fragen).

\hypertarget{hinweise-des-autors}{%
\subsubsection{8. Hinweise des Autors}\label{hinweise-des-autors}}

Dieser Bericht mit allen Berechnungen sind auf
\href{https://github.com/GAEgeler/tilldata_2017}{github} verfügbar.

\hypertarget{quellen}{%
\subsubsection{9. Quellen}\label{quellen}}


\end{document}
